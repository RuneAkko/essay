% !TEX program = latexmk
% !TEX encoding = UTF-8 Unicode

% 注意:请使用 TeXLive 2020 及以上版本,已在 2020 2021 版本测试编译成功,已在 2019 测试编译失败
% 如不想更新,可尝试使用 手动编译 XeLaTeX(可能出错)
% WARNING: Please use TeXLive version >= 2020, This template has been tested and found available on 2020, 2021, unavailable on 2019. 

% 注意:请确保自己已经完整阅读了 README.md 这一 markdown 文档,部分功能的使用方法并未在 .tex 中直接给出;如果仍有使用问题,请在 Github 上提出 issue
% WARNING: Pleans read README.md first cause some usages are not given in .tex files.

%%%%%%%%%%%%%%%%%%%%%%%%%%%%%%%%%%%%%%%%%%%%%%%%%%%%%%%%%%%%%%%%%%%%%%%%%%%%%%%%%%%%%%%%%%%%%%%%%%%%%
%% select the basic style of this thesis/dissertation (this document will use thesis for convenience)
%% 选择论文的基本类型

\documentclass[
    master,     % 必选项:   {master, doctor} 此处不区分专业/学术学位,在下面学位类型处区分
                % Mandatory: {master, doctor} No difference between Academic degree and Professional degree,
                %                             but be careful about arguments in `\degree' and `\subject'
    % english,  % 可选项:   英文正文请选择此项
                % Optional:  For english main content, It will change some auto-generated matter into English
    % blind,    % 可选项:   论文用于盲审请选择此项,将不会产生致谢(内容)、答辩委员会会议决议(内容)、
                %            常规评阅人名单(内容),注意请自行修改 `攻读学位期间取得的研究成果' 内的内容格式
                % Optional:  For blind review, It will not generate Acknowledgements(Content), Decision of Defense Committee(Content),
                %            General Reviewers List(Content), CHANGE THE STYLE of Achievements BY YOURSELF
]{XJTU-thesis}



%%%%%%%%%%%%%%%%%%%%%%%%%%%%%%%%%%%%%%%%%%%%%%%%%
%% fill the each blank for auto-generate contents
%% 填写以下信息用以自动生成

% 论文标题(不超过35个字,英文注意大小写规律)
% Title, Make sure you have an acceptable capitalization
\title{一种基于WebRTC的超低延时直播流传输优化方法}{An Optimization Method Of Ultra-Low Latency Live Streaming Based On WebRTC}

% 学位类型
% 考虑到专业学位的学位名称的少数特例(如专业型法学硕士不是 Master of Juris 而是 Juris Master),此处学位类型请按照案例和文件填写
% 同时为了前向兼容,可选项「A/P」代表「学术/专业」型学位,默认为学术型
% Type of your degree, Translate it from documents in `Materials/Requirements/2021/01 中英文题名页示例/英文标准翻译/'
% \degree[A]{硕士}{Master of Engineering} % 学术型(Academic)硕士请基于 '学术学位名称.txt' 填写
\degree[P]{工程硕士}{Master of Engineering} % 专业型(Professional)硕士请基于 '专业学位(领域)英文标准翻译.pdf' 填写
%\degree[A]{博士}{Doctor of Philosophy}  % 学术型(Academic Doc)博士请填写 '{博士}{Philosophy}'
% \degree[P]{工程博士}{Doctor of Engineering} % 专业型(Professional Doc)博士请基于 '专业学位(领域)英文标准翻译.pdf' 填写

% 作者姓名(注意:所有人名英文均为「名在前,姓在后」,如果只有外文名,请在两个参数都填写外文名称)
% If you have only foreign name, put it as both first and second argument
\author{马瀚森}{Hansen Ma}

% 指导老师姓名(注意:同作者姓名)
% Name of supervisor, It have the same requirments as the author
\advisor{王志文}{副教授}{Zhiwen Wang}{Asscociate Prof.}

% 合作指导老师姓名 或 老师团队(合作指导老师指:1.与招生简章中一致的合作导师,2.CSC项目的合作导师)
%(校方模板要求只能选择一个,都有则显示合作导师)
% Name of associate advisor or adviror's team, You can use only one of them, and advisorassociate has a higher priority.
% \advisorassociate{陈尘}{副教授}{Chen Chen}{Asscociate Prof.}
% \advisorteam{团队中文名}{English Name of the Team}
\advisorteam{陕西省天地网技术重点实验室}{Shaanxi Key Lab. of Satellite-Terrestrial Network Tech. R&D}
% 学科名称,请基于 '学科(专业)英文标准翻译.pdf' 填写
% Name of the subject, also get it from that file
% \subject{航空宇航科学与技术}{Aeronautical and Astronautical Science and Technology}
\subject{计算机技术}{Computer Technology}

% 答辩委员会委员 显示的顺序和这里的一样 第一个人是主席
% Committee member of your defence, notice that the order shows in the thesis is same as here, and the first one is the chairperson
% each member should be put as {Organization,Name,Title} split by comma
\addcommitteemember{西安交通大学,张长长,教授}
\addcommitteemember{西安理工大学,王旺旺,教授}
\addcommitteemember{国网陕西经济技术研究院,李力,高工}
\addcommitteemember{西安交通大学,东方不败,副教授}
\addcommitteemember{西安交通大学,赵照,研究员}

% 答辩时间(手动指定)
% Defence date, input manually
\defensedate{2021}{06}{22}

% 答辩地点(涉密论文请手动设置为「西安交通大学」)
% Defence location, default value is 「西安交通大学」
\defenseloc{西安交通大学主楼E座303室}

% 论文提交日期,不输入参数则默认使用当前日期,如手动指明年月,请在第一个可选参数内填写年份,第二个可选参数填写月份(均为阿拉伯数字)
% Submission date of this thesis, if you not put it manually, it will use the current time
% \submitdate[2021][06]
\submitdate

% 常规审阅人 要求和答辩委员会委员一样
% General reviewer list, same requirements as the committee member
\addgeneralreviewer{西安交通大学,张长长,教授}
\addgeneralreviewer{西安理工大学,王旺旺,教授}
\addgeneralreviewer{国网陕西经济技术研究院,李力,高工}
\addgeneralreviewer{西安交通大学,东方不败,副教授}
\addgeneralreviewer{西安交通大学,赵照,研究员}

% 学院
% School or Faculty, unused now
% \school{电气工程学院}{School of Electrical Engineering}
\school{计算机学院}{School of Computing}
% 专业[学士学位使用]
% Major, unused now
% \stumajor{计算机科学与计数}{Computer Science}

% 学号[学士学位使用]
% Student ID, unused now
% \stuid{}

% 班级[学士学位使用]
% Administrative class, unused now
% \adminclass{电气7xx班}{}

% 参考文献源 参数中不要添加 .bib
% 请使用 \addreferenceresource 添加数据库(可导入多个参考文献数据库)
% 若自动化导入攻读学位期间的成果则使用 \addachivementresource
\addreferenceresource{References/reference}
\addreferenceresource{References/reference}
\addachivementresource{References/achievement}

%%%%%%%%%%%%%%%%%%%%%%%%%%%%%%%%%%%%%%%%%%%%
%% 如果有使用其他包,请在这里添加
%% If you need other packages, use them here

% \usepackage{}


%%%%%%%%%%%%%%%%%%%%%%%%%%%%%%%%%%%%%%%%%%%%%%%%%%%%%%%%%%%%%%%%%%%%%%%%%%%%%%%%%%%
%% 注意:根据校方要求,以下所有页面顺序不可调整
%% Notice: The order of these pages are defined in the requirements of the University.

%% 但在最终提交前,可以通过注释所有 \thesis 开始的命令设置是否生成各个部分
%% 或根据说明调整 latexmkrc 文件,使用 \includeonly 命令导入部分章节。

% \includeonly{
%   Main_Spine/c1,
%   Main_Spine/c2,
%   Main_Spine/c3,
%   Main_Spine/c4,
%   Main_Spine/c5,
%   Main_Spine/c6,
% }

\begin{document}
% [自动生成] 中英题名页
% [Auto Generate] Chinese English Title Page
\thesistitles

% [自动生成] 答辩委员会页
% [Auto Generate] Defense Committee Pages
\thesiscommittes

% 生成摘要页 修改 Main_Miscellaneous/abstract_chs/eng.tex 中的内容
% Abstract, Rewrite your content in Main_Miscellaneous/abstract_chs/eng.tex
\thesisabstract

% [自动生成] 中英目录页
% [Auto Generate] Table of Contents
\thesistableofcontens

% 主要符号表 修改 Main_Miscellaneous/glossary.tex 中的内容
% Glossaries Page, Rewrite your content in Main_Miscellaneous/glossary.tex
\thesisglossarylist

% 正文 注意英文正文写作时,中、英标题还是先中后英标题;同时,下面参数的顺序有意义,不要乱放
% Main contents, X of cX is the chapter of the thesis, you can change it if you want, the order matters
% and KEEP Chinese Title as the FIRST argument, English Titile as the SECOND, if you use `english' option

% 可以通过 \thesisbody 直接导入各部分正文,但也可通过 \thesisbodybegin & \include & \thesisbodyend 组合导入正文

% \thesisbody{
%     Main_Spine/c1,
%     Main_Spine/c2,
%     Main_Spine/c3,
%     Main_Spine/c4,
%     Main_Spine/c5,
%     Main_Spine/c6
% }

\thesisbodybegin
\xchapter{绪论}{Introductions}

\xsection{选题意义和应用背景}{Backgrounds}
随着智能手机等终端设备的普及和网络基础设施的完善,网络直播逐渐成为一种崭新的流媒体应用,过去10年来,广泛应用于娱乐、游戏等行业。随着2020年新冠肺炎疫情大流行,直播行业进一步发展,正在电商、教育、文化传播等众多领域落地。

传统直播传输技术,如[place holder],有高达数十秒的延时。[],将延时降到了3-5秒,但仍难以满足体育直播,课堂互动等业务的低延时需求。2020年以来,疫情进一步促使国内各大头部云厂商行业探索借助开放的 WebRTC 技术,对传统直播 CDN 架构进行优化改造,演化出基于WebRTC的超低延时直播服务,如腾讯云快直播、阿里云 GRTN、字节跳动火山引擎视频云等。WebRTC 是由谷歌开源,为浏览器和移动设备提供实时音视频通讯的开源项目。该技术栈通过基于 UDP 传输协议的 RTP/RTCP 流媒体拓展协议和专为低延时通讯设计的带宽估计,拥塞控制算法,提供了毫秒级的流媒体传输延迟。2021年1月,WebRTC 纳为 W3C 和 IETF 的正式标准\cite{1220034}。2022年3月,火山引擎、阿里云和腾讯云联合发布了“超低延时直播协议信令标准”,标志着基于 WebRTC 的超低延时直播技术在工业界的持续落地。

基于 WebRTC 的超低延时直播,是通过复用 WebRTC 规范在网络传输层和实时编解码方面的技术,来进行延时低于500ms的直播流媒体分发。传统直播流媒体协议[]均是在 TCP 之上设计的,TCP 作为一种可靠的传输层协议,其丢包重传、有序接受、基于丢包的拥塞控制(在绝大部分 Linux 发行版中默认)等特性,引入了不必要的延时,不适合作为实时流媒体传输的传输层协议。WebRTC 使用 UDP 之上的 RTP/RTCP 协议,把拥塞控制工作留给了应用层实现,天然具备更快的发送接收速度,节省了媒体流处理时间,无连接特性也降低了初始延时。[]提出,并在[]上实装的专为实时音视频通讯设计的 WebRTC GCC 带宽估计拥塞控制算法,使用队列延时变化率作为控制信号,对于网络延时变化更为敏感;同时避免了在收发两端维护对抗网络抖动的缓存,进一步降低了延迟。

WebRTC 最初被设计为用于低延时的点对点通讯,在直播场景仍面临诸多挑战。原生的 WebRTC 为了保证双方乃至多方交流的通畅,更倾向在网络不佳下保证低延时,这导致了其在发生丢包场景下估计带宽后撤激进,而在网络恢复后,带宽探测又过于保守;在网络稳定时,其吞吐量过低,带宽利用率不足。在要求高码率、高画质的直播场景中,上述特性导致了用户体验质量的下降。【插入自主实验,量化结果证明】

\xsection{研究内容}{Contents}
本论文为改善基于WebRTC低延迟直播的用户体验,围绕基于传统规则和基于深度强化学习策略两种路径展开研究,提出了一种混合固定规则和启发式策略的带宽估计与拥塞控制机制。经过模拟实验和实机部署测试,该方法在多种网络环境中,在保证低延时的同时,提升了带宽利用率,有效改善了低延迟高清直播中的用户体验。【插入数据】本文的工作内容包括:
\begin{enumerate}[wide,]
    \item 改进的基于规则的 GCC 拥塞控制算法。
    \item 基于深度强化学习的启发式算法。
\end{enumerate}
草稿:native gcc

基于单向队列时延梯度的码率控制机制

第一部分 到达时间模型

概况

使用单向队列时延梯度作为拥塞控制的信号量,可以在不引入过大队列缓存的情况下,进行拥塞控制和带宽估计。基于单向队列时延梯度的拥塞控制机制,要求高精度的时延测量与估计。单向队列时延梯度,也就是极小间隔内的单向队列时延变化率。但队列时延是无法直接测量的。数据的发送和到达时间差,表征了总的传输时延$D_{total}$,它一般包括传播时延$D_{trans}$、队列时延$D_{q}$和网络抖动时延$D_{j}$:

\begin{equation*}
D_{total} = D_{trans} + D_{q} + D_{j}
\end{equation*}

设第$i$、$i+1$步的rtp包的发送时间戳和到达时间戳分别为:$T_{departure}(i)$,$T_{arrival}(i)$和$T_{departure}(i+1)$,$T_{arrival}(i+1)$,显然相邻时间步的$D_{trans}$相同,则有:
\begin{align*}
D_{total}(i+1)-D_{total}(i)
& = D_{trans}(i+1) + D_{q}(i+1) + D_{j}(i+1) - D_{trans}(i)  - D_{q}(i) - D_{j}(i) \\
& = \Delta{D_{q}} +\Delta{D_{j}}
\end{align*}

可以认为$D_{total}(i+1)-D_{total}(i)$是$\Delta{D_{q}}$的测量值,记为$M_{D}$,$\Delta{D_{j}}$是系统噪声。为了最大限度排除其他时延噪声干扰,得到精确的队列时延变化率,可以进行系统建模,使用滤波器来计算$\Delta{D_{q}}$的最佳估计值$\hat{\Delta{D_{q}}}$。

原始算法的滤波器使用有卡尔曼滤波和最小二乘法线性回归估计,为了更精确地进行滤波,本文使用拓展卡尔曼滤波,因为实际系统更接近时变非线性系统。


预过滤器:划分包组

因为流媒体协议RTP/RTCP的传输层是udp协议,因而WebRTC在发送包时,使用平滑发包机制(pacer)。按草案\parencite{rmcat-gcc}建议,一般将$5ms$内封装好的rtp包同时发出,形成规律的脉冲,这种间隔即burst interval。因而,发送时间在$5ms$内的包,可以看作一个包组,这也是测量时延的最小粒度。

除了平滑发包之外,网络链路的瞬时中断(outage)也会影响rtp包的时延。瞬时中断造成过量rtp包堆积在发送端缓存中,并在后续网络恢复后,同时被发送。显然,这种情况影响下的rtp包也应看作一个包组,因为他们的实际时延是类似的。具体而言,在同时满足
\begin{enumerate}
    \item 当前包与上一个包的到达时间间隔小于burst interval;
    \item 当前包与当前组的最新包(即上一个包)相比,时延减小;
\end{enumerate}
两个条件下,也认为属于同一个包组。

滤波器估计队列时延变化率

卡尔曼滤波器

最小二乘线性回归

在MXX之后的Chrome版本中,WebRTC使用了最小二乘法,即使用最小误差平方和,来对累计时延-时间曲线进行线性回归,将得到的直线方程斜率,作为队列时延变化率。该算法部署在发送端,因而不能直接通过接收到的RTP包的统计情况来计算样本点。由草案规定的[]RTCP feedback报文,来进行采样、线性回归。

时延-时间曲线的样本收集粒度为RTP包级。对于时间轴坐标,零点设为整个媒体传输过程,第一个RTP包的到达时间。对于累计时延轴,每个RTP包样本点使用所在RTP包组的累计时延作为纵轴坐标。线性回归以固定长度的滑动窗口为尺度进行计算,长度单位是收到的RTP包个数,现有代码中设置为20。






拓展卡尔曼滤波


基于丢包的码率控制

基于单向队列时延梯度的码率控制机制,在

\begin{enumerate}
    \item 链路中队列缓存过小时,过载探测器将无法探测单向队列时延梯度;
    \item 与其他基于丢包的拥塞控制算法共存,WebRTC流会陷入饥饿;
\end{enumerate}

两种情况下将失灵。由此引入基于丢包的码率控制模块,作为兜底。

由草案\parencite{rmcat-gcc}规定,位于发送端的丢包控制器依据草案\parencite{twcc}约定的transport-wide feedback message报文得到接收方收到的sequence number,自行计算丢包率。该报文一般每个视频帧回传一次(每30ms-100ms回传一次);或依据草案\parencite{rmcat-remb},通过接收方回传的RTCP Receiver Report报文,得到丢包率。

设丢包率为$Loss$,估计发送码率为$A_s$,控制间隔为$i$。有以下算法:

\begin{equation*}
A_s(i) = \begin{cases}
    A_s(i-1) * (1- 0.5*Loss),& Loss > 0.1\\
    1.05 * A_s(i-1),& Loss < 0.02\\
    A_s(i-1), & \text{otherwise}
\end{cases}
\end{equation*}

在丢包率低于$2\%$时,认为是与流自身拥塞无关的偶然丢包,忽略并乘性增加估计码率。如果拥塞丢包存在,那么丢包率会很快超过$10\%$,此时减少估计码率。








% \include{Main_Spine/c1}
% \include{Main_Spine/c2}
% \include{Main_Spine/c3}
% \include{Main_Spine/c4}
% \include{Main_Spine/c5}
% \include{Main_Spine/c6}
\thesisbodyend

% 致谢 修改 Main_Miscellaneous/acknowledegment.tex 中的内容
% Acknowledgement, Rewrite your content in Main_Miscellaneous/acknowledegment.tex
\thesisacknowledegment

% [自动生成] 参考文献 默认使用 References/reference.bib,手动指定请在导言区 \addbibresource 处指定
% [Auto Generate] Bibliography, Default file is References/reference.bib, change argument in \addbibresource for manual specification
\thesisbibliography

% 附录(有几个附录就导入几个文件(不加.tex后缀)),
% Appendi(x/ies), argument should not have the .tex suffix
\thesisappendix{Main_Miscellaneous/appendix_a,
                Main_Miscellaneous/appendix_b}

% 攻读学位期间取得的研究成果
% 添加 [auto] 参数则读取通过 \addachivementresource 添加的数据库,否则读取 Main_Miscellaneous/achievement.tex 的内容 注意盲审时需要手动修改格式
% 自动读取的数据库中 若条目含有 AUTHOR+an = {X=highlight} 则第 X 位作者会被加粗
% Achievement, argument [auto] will load data from database added by \addachivementresource, or the data from Main_Miscellaneous/achievement.tex
% \thesisachivements
\thesisachivements[auto]

% 答辩委员会决议 修改 Main_Miscellaneous/decision.tex 中的内容
% Decision, Rewrite your content in Main_Miscellaneous/decision.tex
\thesisdecision

% [自动生成] 常规评阅人名单 需要手动指定两个数字作为'本学位论文共接受 {#1} 位专家评阅,其中常规评阅人 {2}名'内容参数
% [Auto Generate] Reviewer List, set two number as the content in this page 
\thesisreviewers{7}{5}

% [自动生成] 独创性声明
% [Auto Generate] Originality Declaration
\thesisdeclarations

\end{document}