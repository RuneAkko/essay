\xchapter{绪论}{Introductions}

\xsection{选题意义和应用背景}{Backgrounds}
随着智能手机等终端设备的普及和网络基础设施的完善,网络直播逐渐成为一种崭新的流媒体应用,过去10年来,广泛应用于娱乐、游戏等行业。随着2020年新冠肺炎疫情大流行,直播行业进一步发展,正在电商、教育、文化传播等众多领域落地。

传统直播传输技术,如[place holder],有高达数十秒的延时。[],将延时降到了3-5秒,但仍难以满足体育直播,课堂互动等业务的低延时需求。2020年以来,疫情进一步促使国内各大头部云厂商行业探索借助开放的 WebRTC 技术,对传统直播 CDN 架构进行优化改造,演化出基于WebRTC的超低延时直播服务,如腾讯云快直播、阿里云 GRTN、字节跳动火山引擎视频云等。WebRTC 是由谷歌开源,为浏览器和移动设备提供实时音视频通讯的开源项目。该技术栈通过基于 UDP 传输协议的 RTP/RTCP 流媒体拓展协议和专为低延时通讯设计的带宽估计,拥塞控制算法,提供了毫秒级的流媒体传输延迟。2021年1月,WebRTC 纳为 W3C 和 IETF 的正式标准\cite{1220034}。2022年3月,火山引擎、阿里云和腾讯云联合发布了“超低延时直播协议信令标准”,标志着基于 WebRTC 的超低延时直播技术在工业界的持续落地。

基于 WebRTC 的超低延时直播,是通过复用 WebRTC 规范在网络传输层和实时编解码方面的技术,来进行延时低于500ms的直播流媒体分发。传统直播流媒体协议[]均是在 TCP 之上设计的,TCP 作为一种可靠的传输层协议,其丢包重传、有序接受、基于丢包的拥塞控制(在绝大部分 Linux 发行版中默认)等特性,引入了不必要的延时,不适合作为实时流媒体传输的传输层协议。WebRTC 使用 UDP 之上的 RTP/RTCP 协议,把拥塞控制工作留给了应用层实现,天然具备更快的发送接收速度,节省了媒体流处理时间,无连接特性也降低了初始延时。[]提出,并在[]上实装的专为实时音视频通讯设计的 WebRTC GCC 带宽估计拥塞控制算法,使用队列延时变化率作为控制信号,对于网络延时变化更为敏感;同时避免了在收发两端维护对抗网络抖动的缓存,进一步降低了延迟。

WebRTC 最初被设计为用于低延时的点对点通讯,在直播场景仍面临诸多挑战。原生的 WebRTC 为了保证双方乃至多方交流的通畅,更倾向在网络不佳下保证低延时,这导致了其在发生丢包场景下估计带宽后撤激进,而在网络恢复后,带宽探测又过于保守;在网络稳定时,其吞吐量过低,带宽利用率不足。在要求高码率、高画质的直播场景中,上述特性导致了用户体验质量的下降。【插入自主实验,量化结果证明】

\xsection{研究内容}{Contents}
本论文为改善基于WebRTC低延迟直播的用户体验,围绕基于传统规则和基于深度强化学习策略两种路径展开研究,提出了一种混合固定规则和启发式策略的带宽估计与拥塞控制机制。经过模拟实验和实机部署测试,该方法在多种网络环境中,在保证低延时的同时,提升了带宽利用率,有效改善了低延迟高清直播中的用户体验。【插入数据】本文的工作内容包括:
\begin{enumerate}[wide,]
    \item 改进的基于规则的 GCC 拥塞控制算法。
    \item 基于深度强化学习的启发式算法。
\end{enumerate}